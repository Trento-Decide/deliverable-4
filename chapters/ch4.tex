\chapter{Criticità}

Durante lo sviluppo abbiamo affrontato alcune sfide, dovute soprattutto al fatto che il progetto non riguardava solo la scrittura del codice. In particolare, abbiamo dovuto dedicare molta attenzione a due aspetti: la revisione della parte progettuale e la ricerca di un equilibrio tra realismo e fattibilità.

\section{Verifica delle componenti}

Mentre controllare se il codice funziona è abbastanza immediato, verificare la correttezza di diagrammi UML, casi d'uso o requisiti è più complesso. 

Trattandosi di elementi che lasciano spazio all'interpretazione, è stato necessario un continuo lavoro di revisione e confronto interno. Questo non ha creato problemi organizzativi, ma ha richiesto molto tempo e attenzione, impegnando in modo particolare il team leader nel gestire le diverse versioni e garantire la coerenza dei documenti.

\section{Coerenza con la realtà}

Un'altra sfida è stata scegliere funzionalità che fossero credibili per una piattaforma reale, ma allo stesso tempo adatte a un progetto universitario.

Non avendo un'esperienza professionale specifica nel settore, ci siamo trovati spesso a discutere diverse alternative. Trovare ogni volta un punto d'incontro tra idee differenti ha richiesto molto tempo.

\subsection{Esempio: La Moderazione} 
Abbiamo discusso a lungo sulla necessità di inserire un sistema di moderazione. Da un lato, per un progetto d'esame, sviluppare questa funzione poteva sembrare un lavoro eccessivo e non strettamente richiesto. Dall’altro, per rendere il prodotto credibile e realistico, la moderazione era un elemento che non potevamo ignorare. Trovare un punto d'incontro tra semplicità didattica e realismo ha richiesto un acceso confronto, portandoci a scegliere una soluzione pragmatica che non appesantisse troppo il lavoro ma garantisse la serietà della piattaforma.
