\chapter{Criticità}
Da settembre a oggi, posso definire i due problemi che, secondo me, sono intrinseci nella struttura dell’esame: quanto sia sufficiente chiudere concettualmente, dal punto di vista commerciale, il nostro prodotto, e come verificare la correttezza di tutto ciò che non è software.

\section{Verifica delle componenti}
Verificare che il software sia corretto è banale; verificare che un diagramma, che dà molto spazio all’interpretabilità, sia perfettamente corretto e coerente è, secondo noi, estremamente più complicato. Proprio per questo, io stesso ho dovuto dedicare larghissime fasce temporali alla sola verifica e alla concentrazione nell’individuazione delle diverse interpretazioni possibili di un dato oggetto.
Tirando le somme riguardo a questa problematica, essa non ha creato dissapori all’interno del team, ha solamente necessitato di molto tempo.

\section{Coerenza con la realtà}
La criticità più rilevante, che durante il mese di novembre ha generato diverse discussioni tra me e gli altri due membri del gruppo, ha riguardato le decisioni legate alla “user experience” e, più in generale, al realismo del prodotto. Nessuno di noi possedeva un'esperienza pregressa tale da poter definire con certezza inoppugnabile quale fosse la scelta corretta in ogni situazione; di conseguenza, ci siamo trovati spesso a dover mediare tra visioni divergenti. Di seguito riporto un esempio emblematico per spiegarmi meglio:

\subsection{Esempio: La Moderazione} 
Ci siamo scontrati a lungo sulla necessità o meno di progettare un sistema di moderazione dei contenuti. Da un lato, trattandosi di un progetto accademico e didattico, dedicare molto tempo a una funzionalità simile poteva sembrare uno sforzo sproporzionato e non strettamente necessario ai fini dell'esame. Dall'altro lato, però, l'obiettivo era mantenere una rigorosa "coerenza con la realtà": per simulare un prodotto commerciale che fosse davvero credibile e utilizzabile da utenti reali, era impensabile tralasciare un aspetto critico come la moderazione. Trovare un equilibrio tra la semplificazione didattica e il realismo commerciale ha richiesto molto tempo e un acceso confronto.