\chapter{Organizzazione del lavoro}

\section{Introduzione ai ruoli}

L'idea fondante riguardo alla suddivisione del lavoro è stata che ogni membro del gruppo dovesse occuparsi a tutto tondo di ogni aspetto del progetto. Ho dunque deciso volutamente di non suddividere rigidamente i compiti (ad esempio, un membro solo al backend, un altro al frontend, un altro a tutto il D3); ognuno ha toccato con mano tutti i deliverable, in proporzioni maggiori o minori. 
\\
Sono perfettamente consapevole che, per questioni puramente temporali e tecniche, questo approccio possa risultare inefficiente e rischioso: in ambito aziendale, infatti, è assolutamente necessaria una rigida suddivisione. Tuttavia, in questo contesto, ho ritenuto che un approccio del genere sarebbe stato troppo povero dal punto di vista didattico.
\\
Alla luce di ciò, l'unica vera distinzione di ruolo ha riguardato la figura del manager: quest'ultimo, oltre a svolgere operativamente il medesimo lavoro degli altri membri del gruppo, si è occupato di supervisionare ad alto livello l'avanzamento e l'organicità del nostro prodotto.

\section{Procedimenti e strumenti organizzativi}
\subsection{Procedimenti}
Per quanto riguarda l'organizzazione, di settimana in settimana ho stilato una lista di task da portare a termine, indicando il nome del membro del team incaricato e aggiungendo eventuali note o risorse utili per lo sviluppo del compito assegnato.
Al termine della settimana lavorativa organizzavamo delle brevi riunioni, di un'ora al massimo, in cui commentavamo il lavoro svolto e facevamo "il punto della situazione".

\subsection{Strumenti}
Per la stesura dei documenti abbiamo utilizzato LaTeX; il codice sorgente di tutti i file è interamente disponibile nei vari repository su GitHub al seguente link: \url{https://github.com/Trento-Decide/}.
Per quanto riguarda invece il documento relativo alla suddivisione oraria, abbiamo utilizzato un semplice foglio Excel condiviso in cloud.