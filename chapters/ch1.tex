\chapter{Organizzazione del lavoro}

\section{Introduzione ai ruoli}

Abbiamo scelto di non dividerci i compiti in modo rigido: l'obiettivo era che ogni membro avesse una visione d'insieme del progetto, senza limitarsi a una singola parte. 

Invece di assegnare ruoli fissi (ad esempio un responsabile esclusivo per il backend o per la documentazione), abbiamo adottato un approccio trasversale. Ognuno ha contribuito a tutte le fasi del lavoro in misure quanto più simili, assicurando così che l'intero gruppo avesse sempre una conoscenza condivisa delle scelte fatte.

\vspace{0.3cm}

Siamo consapevoli che questa scelta possa risultare meno efficiente rispetto a una specializzazione netta, che in ambito aziendale è spesso necessaria per massimizzare la produttività. Tuttavia, in un contesto universitario, abbiamo privilegiato l'aspetto didattico: una divisione troppo settoriale avrebbe limitato la comprensione dell'architettura generale e del sistema nella sua interezza.

\vspace{0.3cm}

Una volta definiti i confini del progetto, abbiamo lavorato su più fronti in parallelo. Questa strategia ci ha permesso di ottimizzare i tempi, evitare colli di bottiglia tra una fase e l'altra e rispettare le scadenze dei vari deliverable.

\vspace{0.3cm}

L'unica distinzione di ruolo ha riguardato il team leader che, oltre al normale lavoro di sviluppo insieme agli altri, si è occupato di coordinare il gruppo, monitorare l'avanzamento dei compiti e garantire che il progetto finale fosse coerente.


\section{Procedimenti e strumenti organizzativi}

\subsection{Procedimenti}

Di settimana in settimana, il team leader ha preparato una lista di task da portare a termine, assegnandoli ai membri del gruppo con le indicazioni e le risorse necessarie. 

Abbiamo pianificato le attività in modo incrementale, cercando di mantenere un equilibrio tra lo sviluppo tecnico e la stesura della documentazione.

A fine settimana ci riunivamo per una breve riunione, di massimo un'ora, per valutare il lavoro svolto, risolvere eventuali criticità e pianificare gli obiettivi successivi. Questo metodo ci ha permesso di intervenire subito in caso di ritardi o problemi imprevisti.

\subsection{Strumenti}

Per i documenti abbiamo utilizzato \LaTeX, così da ottenere una documentazione modulare e uniforme nello stile.

Il codice sorgente è disponibile sui repository GitHub al seguente link:

\begin{center}
\url{https://github.com/Trento-Decide/}
\end{center}

Per gestire i tempi e l'organizzazione interna, abbiamo usato un semplice spreadsheet condiviso in cloud. Questo strumento è stato sufficiente per tracciare il contributo di ognuno e l'impegno richiesto dalle diverse fasi del progetto in totale trasparenza.